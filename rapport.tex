\documentclass[12pt,oneside,a4paper]{article}

%% Language and font encodings
\usepackage{ulem}
\usepackage[danish,english]{babel}
\usepackage[utf8x]{inputenc}
\usepackage[T1]{fontenc}

%% Sets page size and margins
\usepackage[left=2.5cm,top=2.0cm,bottom=1.5cm, right=3.0cm]{geometry}

%% Useful packages
\usepackage{amsmath}
\usepackage{graphicx}
\usepackage{epsfig}
\usepackage[colorinlistoftodos]{todonotes}
\usepackage[colorlinks=true, allcolors=blue]{hyperref}
\usepackage{gensymb}
\usepackage{float}
\usepackage{adjustbox}
\usepackage{graphics}
\usepackage{amssymb}
\usepackage{url}
\usepackage{cancel}
\usepackage{booktabs}
\usepackage{pdfpages}
\usepackage{mathpazo}
\usepackage[section]{placeins}
\usepackage{caption}
\usepackage{wrapfig}
\usepackage{tocloft}
\usepackage{subfig}
\usepackage{fancyhdr}
\usepackage{framed}
\usepackage{footnote}

\linespread{1.3} % linjeafstand 1.5
\setlength{\parindent}{0cm}
\setlength{\parskip}{0.3cm}

% Title Page
%\title{Simulering af neutron optik}
%\author{
%    Hyatt, Jonas Peter\ 
%    \texttt{zkv499@alumni.ku.dk}
%    \and
%    LastName, Walde\ 
%    \texttt{first2.last2@alumni.ku.dk}
%    \and
%    LastName, Walther\ 
%    \texttt{first2.last2@alumni.ku.dk}
%}


\begin{document}
\selectlanguage{danish}
\pagenumbering{roman}
%\maketitle

%% Forside %%

\begin{center}
    {\textsc {\LARGE \bf{Københavns Universitet \\[0.3cm]  Bachelorstudiet i fysik}}}\\[1.5cm]
    {\textsc {\Large \bf Førsteårsprojekt 2017}}\\[0.8cm]
    {\Large Projekt nummer: 2017-05}\\[1cm]
    
    \rule{15cm}{0.01cm}\\[1cm]
    {\LARGE\bf  Simulering af neutron optik} \\ {\it Med fokus på M-Optimization}\\ [0.5cm]
    \rule{15cm}{0.01cm}\\[1cm]
\end{center}

\vfill
{\large Forfattere:}\\
{\large \hspace*{1cm} \makebox[6cm][l]{Jonas P. Hyatt}  \hspace{1cm} KU- ID: \makebox[2cm][l]{ZKV499} \\
{\large \hspace*{1cm} \makebox[6cm][l]{Waldemar Svejstrup}   \hspace{1cm} KU- ID: \makebox[2cm][l]{MDS274} \\
{\large \hspace*{1cm} \makebox[6cm][l]{Walter J. Birkemose}   \hspace{1cm} KU- ID: \makebox[2cm][l]{SMX359} \\ 
        
{\large Vejledere:}\\
{\large \hspace*{1cm} \makebox[6cm][l]{Kim Lefmann}  \hspace{1cm} Email: \makebox[2cm][l]{lefmann@nbi.ku.dk} \\
{\large \hspace*{1cm} \makebox[6cm][l]{Jonas Okkels Birk}    \hspace{1cm} Email: \makebox[2cm][l]{jonasobirk@gmail.com} \\

\vfill

{\large Rapporten omfatter {\bf 1} siders hovedtekst og {\bf 1} siders appendix.}

{\large Rapporten er indsendt som en pdf-fil den 17 marts 2017. }

\normalsize

%% Forside slut! %%
\newpage
%% Abstract %%

\begin{abstract}
    Vi simulerer små bolde - Jeg tænker vi kalder det abstract istedet for resumé men kunne ikke finde hvor man ændrer det :))
Vi skriver abstract på engelsk
    
    We are simulating small balls
\end{abstract}

%% Abstract slut! %%
\newpage
%% Indholdsfortegnelse %%

\tableofcontents

%% Indholdsfortegnelse slut %%
\newpage
%% Rapporten begynder her! %%

\pagenumbering{arabic}

\section{Introduktion}

Vores samfund har over de sidste par hundrede år, været under en rivende udvkiling. Mange nye opfindelser er blevet fundet, og mange gamle opfindelser er blevet optimeret. I vores søgen efter mere, leder vi konstant efter nye og bedre måder, at gøre tingene på. På trods af den teknologiske udvikling, har vi stadig mange udfordringer tilbage. Dette kan være alt fra miljø- og klima problemer, til problemer inden for sundhed. For at udvikle og studere de materialer vores verden består af, er det vigtigt for forskerne at kunne se hvordan alting er opbygget. Et vigtigt redskab, der giver forskerne mulighed for at studere materialer helt ned til atomart niveau, er neutron spredning. Dette er et redskab der giver interessante resultater ikke blot for fysik, men også for eksempelvis kemi og bioteknologi, da forståelse af materialers opbygning, er essentiel i alle naturvidenskabelige fag. Netop neutron spredning bliver brugt ved ESS (European Spallation Source), i Lund i Sverige.

\subsection{Neutron Spredning}
Her introducerer vi neutron spredning

\subsection{Monte Carlo simuleringer}
Her fortæller vi om Monte Carlo 




\section{Metode}
Her starter vores store metode afsnit

\subsection{MCstas}
Her skriver vi om MCstas, hvordan det fungerer og hvorfor det er godt/skidt

\subsection{Komponenter til guidesystermerne}
Her forklarer vi hvordan vi skruer vores guides sammen. Begreber som chopper, sample, diverse former for monitors forklares osv.

\subsection{Krav til guidesystemerne}
Hvad skal være opfyldt? Penge? Antal neutroner? Divergens? BT? 

\subsection{Bedømmelse af guidesystemer}
Her snakker vi om hvordan vi bedømmer guidesystemerne, dvs. antal neutroner, divergens, brilliance transfer osv.

\subsection{Svagheder og fejlkilder ved MCstas}
Hvorfor er det godt og hvorfor er det skidt?



\section{Resultater}

\subsection{Vores guide(s)}
Plots of beskrivelse af vores guides



\section{Diskussion}

\subsection{Bedømmelse af vores guides}
Her snakker vi om styrker/svagheder ved vores systemer

\section{Konklusion}



%% Rapporten er slut! %% 
\newpage
%% Appendix %%

\appendix
\section{Appendix}
PDF, ESS Gymnasiehæfte

%% Appendix slut! %%

\end{document}

